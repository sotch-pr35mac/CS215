
%% Based on a TeXnicCenter-Template by Tino Weinkauf.
%%%%%%%%%%%%%%%%%%%%%%%%%%%%%%%%%%%%%%%%%%%%%%%%%%%%%%%%%%%%%

%%%%%%%%%%%%%%%%%%%%%%%%%%%%%%%%%%%%%%%%%%%%%%%%%%%%%%%%%%%%%
%% HEADER
%%%%%%%%%%%%%%%%%%%%%%%%%%%%%%%%%%%%%%%%%%%%%%%%%%%%%%%%%%%%%
%\documentclass[onecolumn, 11pt, conference, compsocconf]{IEEEtran}
\documentclass[onecolumn, 12pt, article]{IEEEtran}
\usepackage{times,color,amsmath,amssymb,amsthm,comment,graphicx,cite,multirow}
%\documentclass[letterpaper,oneside,11pt]{letter}
% Alternative Options:
%	Paper Size: a4paper / a5paper / b5paper / letterpaper / legalpaper / executivepaper
% Duplex: oneside / twoside
% Base Font Size: 10pt / 11pt / 12pt


%% Language %%%%%%%%%%%%%%%%%%%%%%%%%%%%%%%%%%%%%%%%%%%%%%%%%
\usepackage[USenglish]{babel} %francais, polish, spanish, ...
\usepackage[T1]{fontenc}
\usepackage[ansinew]{inputenc}

\usepackage{lmodern} %Type1-font for non-english texts and characters
\usepackage{algorithm}
\usepackage{algpseudocode}
\usepackage{float}

%% Packages for Graphics & Figures %%%%%%%%%%%%%%%%%%%%%%%%%%
%\usepackage{graphicx} %%For loading graphic files
%\usepackage{subfig} %%Subfigures inside a figure
%\usepackage{tikz} %%Generate vector graphics from within LaTeX

%% Please note:
%% Images can be included using \includegraphics{filename}
%% resp. using the dialog in the Insert menu.
%% 
%% The mode "LaTeX => PDF" allows the following formats:
%%   .jpg  .png  .pdf  .mps
%% 
%% The modes "LaTeX => DVI", "LaTeX => PS" und "LaTeX => PS => PDF"
%% allow the following formats:
%%   .eps  .ps  .bmp  .pict  .pntg


%% Math Packages %%%%%%%%%%%%%%%%%%%%%%%%%%%%%%%%%%%%%%%%%%%%
\usepackage{amsmath}
\usepackage{amsthm}
\usepackage{amsfonts}
\usepackage{amssymb}
%\usepackage{array,MnSymbol}


%% Line Spacing %%%%%%%%%%%%%%%%%%%%%%%%%%%%%%%%%%%%%%%%%%%%%
\usepackage{setspace}
\singlespacing        %% 1-spacing (default)
%\onehalfspacing       %% 1,5-spacing
%\doublespacing        %% 2-spacing


%% Other Packages %%%%%%%%%%%%%%%%%%%%%%%%%%%%%%%%%%%%%%%%%%%
%\usepackage{a4wide} %%Smaller margins = more text per page.
\usepackage{fancyhdr} %%Fancy headings
\usepackage{listings}
\usepackage{capt-of}


%
% Theorem like environments
%
\newtheorem{problem}{Problem}%
%\numberwithin{problem}{section}
\newtheorem{theorem}{Theorem}%
\newtheorem{acknowledgment}{Acknowledgment}%
%\newtheorem{algorithm}{Algorithm}%
\newtheorem{assumption}{Assumption}%
\newtheorem{axiom}{Axiom}%
\newtheorem{case}{Case}%
\numberwithin{case}{problem}
\newtheorem{claim}{Claim}%
\newtheorem{conclusion}{Conclusion}
\newtheorem{condition}{Condition}
\numberwithin{condition}{problem}
\numberwithin{condition}{subsection}
\newtheorem{conjecture}{Conjecture}
\newtheorem{corollary}{Corollary}
\newtheorem{criterion}{Criterion}
\newtheorem{definition}{Definition}
\numberwithin{definition}{section}
\newtheorem{example}{Example}
\newtheorem{exercise}{Exercise}%
\newtheorem{lemma}{Lemma}%
\newtheorem{notation}{Notation}%
\theoremstyle{remark}
\newtheorem{question}{Question}%
\numberwithin{question}{problem}
\theoremstyle{plain}
\newtheorem{answer}{Answer}%
\numberwithin{answer}{problem}
\newtheorem{proposition}{Proposition}%
\newtheorem{remark}{Remark}%
\newtheorem{solution}{Solution}%
\numberwithin{solution}{section}
\newtheorem{summary}{Summary}%
\numberwithin{equation}{section}%
\newtheorem{option}{Option}



\raggedbottom
%%%%%%%%%%%%%%%%%%%%%%%%%%%%%%%%%%%%%%%%%%%%%%%%%%%%%%%%%%%%%
%% Remarks
%%%%%%%%%%%%%%%%%%%%%%%%%%%%%%%%%%%%%%%%%%%%%%%%%%%%%%%%%%%%%
%
% Note:
% 1. Edit the used packages and their options (see above).
% 2. If you want, add a BibTeX-File to the project
%    (e.g., 'literature.bib').
% 3. Happy TeXing!
%
%%%%%%%%%%%%%%%%%%%%%%%%%%%%%%%%%%%%%%%%%%%%%%%%%%%%%%%%%%%%%

%%%%%%%%%%%%%%%%%%%%%%%%%%%%%%%%%%%%%%%%%%%%%%%%%%%%%%%%%%%%%
%% Options / Modifications
%%%%%%%%%%%%%%%%%%%%%%%%%%%%%%%%%%%%%%%%%%%%%%%%%%%%%%%%%%%%%

%\input{options} %You need a file 'options.tex' for this
%% ==> TeXnicCenter supplies some possible option files
%% ==> with its-templates (File | New from Template...).




%% BEGIN DOCUMENT
\begin{document}

%% Title Page
\title{0/1 Knapsack}
\author{Preston Stosur-Bassett}
\date{March 11, 2015}
\maketitle

\pagestyle{fancy}
\fancyhead[R]{0/1 Knapsack in Java, page \thepage}
\fancyhead[L]{Preston Stosur-Bassett}

%% BEGIN ABSTRACT
\begin{abstract}
%%TODO: write this

%% END ABSTRACT
\end{abstract}

%% BEGIN MOTIVATION
\section{Motivation}
%%TODO: write this

%% END MOTIVATION

%% BEGIN BACKGROUND
\section{Background}
%%TODO: write this

%% END BACKGROUND

%% BEGIN PROCEDURE
\section{Procedure}
%%TODO: write this

%% END PROCEDURE

%% BEGIN TESTING
\section{Testing}
\subsection{Testing Plan and Results}
%%TODO: write this

\subsection{Problems Encountered}
%%TODO: write this

%% END TESTING

%% BEGIN EXPERIMENTAL ANALYSIS
\section{Experimental Analysis}
%%TODO: write this

%% END EXPERIMENTAL ANALYSIS

%% BEGIN CONCLUSION
\section{Conclusion}
%%TODO: write this

%% END CONCLUSION

\newpage

%% BEGIN REFERNCES
\section*{References}

%% \nocite{*}
%% \bibliographystyle{IEEEtran}
%% \bibliography{bib}

%% END REFERENCES

\newpage

%% BEGIN APPENDIX
\section*{Appendix}
\lstinputlisting[caption=Driver,
label={Driver.java},
breaklines=true,
]{Code/Driver.java}
\lstinputlisting[caption=Knapsack,
label={Knapsack.java},
breaklines=true,
]{Code/Knapsack.java}
\lstinputlisting[caption=Sort,
label={Sort.java},
breaklines=true,
]{Code/Sort.java}

The class Stopwatch has been altered from its original form.
\newline
\lstinputlisting[caption=Stopwatch,
label={Stopwatch.java},
breaklines=true,
]{Code/Stopwatch.java}
\lstinputlisting[caption=Debug,
label={Debug.java},
breaklines=true,
]{Code/Debug.java}
\lstinputlisting[caption=DummyData,
label={DummyData.java},
breaklines=true,
]{Code/DummyData.java}
%% END APPENDIX

%% END DOCUMENT
\end{document}
